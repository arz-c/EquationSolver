\documentclass[11pt]{article}
% \usepackage{amsmath}
\usepackage{mymacros}
\usepackage[utf8]{inputenc}
\usepackage[margin=0.75in]{geometry}

\title{CSC111 Project 2 Proposal: Tree-Based Equation Solver}
\author{Areez Chishtie}
\date{\today}

\begin{document}
\maketitle

\section*{Problem Description and Research Question}

There are many tools available online for solving various classes of equations. According to Wikipedia, this kind of software is called a \textit{computer algebra system (CAS)}. The most obvious utility offered by a CAS is to perform and/or verify computations that we may do, but more generally, a CAS provides an interface for representing and manipulating algebraic objects with code. The standard data structure used by CAS's to represent algebraic equations are trees, in a way that is reminiscent of our case study of abstract syntax trees. \textbf{The core of my project will consist of a basic computer algebra system that takes an equation and a desired variable as input, and produces a solution to the equation as output.} The system should be capable of solving linear/quadratic equations, some other polynomial equations, and possibly, some equations involving trigonometric and exponential functions. The project will also include a visualization of the tree induced by the given equation, and hopefully a step-by-step walkthrough of the manipulations leading to the solution.

\section*{Computational Plan}

The CAS will use a tree to represent the given equation: the root values will be unary/binary operations/symbols like $+, -, \times, =, \log$, and the left/right subtrees will be expressions taken as arguments on the left-/right-hand sides of the operation. On this tree, the CAS will use some kind of algorithm to recursively "apply operations to both sides" until the corresponding equation attains a pre-defined form, from which point a base case could explicitly construct the solution(s) (e.g., by applying the quadratic equation).

I intend for the user to be able to type the equation as a string input, so some part of the CAS needs to be responsible for converting a string into a tree. I would like the output to be a graphical user interface that includes an interactable representation of the tree for the user's given equation. A nice feature would be the ability to "walk through" the produced solution and see what the tree looked like at each stage of the algorithm.

The graphical interface will be built with the help of an additional Python library. One promising library is "igraph", which supports the visualization of a tree as a special case of a graph. This library has a "Graph" class which stores a list of tuples representing the edges. Once I compute these tuples, a simple call to the "plot" method, with the layout option set to "Reingold-Tilford", should produce a visualization of the graph as a tree.

\section*{References}

\begin{itemize}
\item Wikipedia contributors. (2023, December 20). Computer algebra system. Wikipedia. \\
    \href{https://en.wikipedia.org/wiki/Computer_algebra_system?oldformat=true}{https://en.wikipedia.org/wiki/Computer_algebra_system?oldformat=true}
\item Solving equations. (n.d.). The University of Texas at Austin, Computer Science. Retrieved March 3, 2024, from \href{https://www.cs.utexas.edu/users/novak/algebra.pdf}{https://www.cs.utexas.edu/users/novak/algebra.pdf}
\item igraph R manual pages. (n.d.). \href{https://igraph.org/r/doc/}{https://igraph.org/r/doc/}
\item Igraph. (n.d.). The nodes of tree are overlap when plot a tree. Maybe a bug when plot compex trees. · Issue \#186 · igraph/python-igraph. GitHub. \href{https://github.com/igraph/python-igraph/issues/186}{https://github.com/igraph/python-igraph/issues/186}
\end{itemize}






% NOTE: LaTeX does have a built-in way of generating references automatically,
% but it's a bit tricky to use so you are allowed to write your references manually
% using a standard academic format like MLA or IEEE.
% See project proposal handout for details.

\end{document}
